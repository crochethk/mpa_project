
\usepackage[utf8]{inputenc} % Ermöglicht die Eingabe von deutschen Sonderzeichen im Quelltext
\usepackage[T1]{fontenc} % Europäische Kodierung der Ausgabeschrift
\usepackage{lmodern} % Lädt die Schriftart Latin Modern, die universell skalierbar ist
\usepackage[ngerman]{babel} % Silbentrennung nach neuer deutscher Rechtschreibung
\usepackage{amsmath} % Mathematik-Paket der American Mathematical Society
\usepackage{graphicx} % Erlaubt das Einfügen von Bildern per \includegraphics{}
\usepackage{float}

%--------------- custom packages
\usepackage{fullpage}
%\usepackage{todonotes}
\usepackage{siunitx} % für Dezimal Trennzeichen

\usepackage{listings}
\lstset{
    frame=single,
    basicstyle=\ttfamily,
    columns=flexible,
    showstringspaces=false,
    breakautoindent=true,
    breaklines=true
}
\usepackage{listings-rust}
\usepackage{xcolor}





\usepackage{dirtree}

% \usepackage{tikz}

% \usepackage{caption}
% \captionsetup{
	%  figurename=Abb.,
	% %  tablename=Tab.
	% }

% Workaround für "--" wird zu "–" trotz \texttt
\usepackage{microtype}
\DisableLigatures{encoding = T1, family = tt*}


\usepackage{hyperref}
%------------------------------------

%--------------- custom Commands
\providecommand{\tightlist}{\setlength{\itemsep}{0pt}\setlength{\parskip}{0pt}}

% Abkürzung für Multiple Precision Integer
\newcommand{\mpi}{\emph{MPI}}

% INLINE code
\newcommand{\ilc}[1]{\texttt{#1}}



