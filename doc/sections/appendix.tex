
\appendix
\section{Operand Scanning Code} \label{apdx:opscan}
\begin{lstlisting}[language=Rust, style=boxed]
    /// Multiplies two `MPint`, extending the width as necessary.
    /// This uses a kind of "Operand-Scanning" algorithm.
    fn extending_operand_scan_mul(&self, rhs: &Self) -> Self {
        self.assert_same_width(rhs);

        // Zero short circuit
        if self.is_zero() {
            return self.clone();
        } else if rhs.is_zero() {
            return rhs.clone();
        }

        let result_sign = if self.sign == rhs.sign {
            // same signs implicate positive result
            Sign::Pos
        } else {
            // different signs implicate negative result
            Sign::Neg
        };
        let max_new_width = self.width * 2;

        // ~~~~ Main ~~~~

        // Given `a*b = a0..an * b0..bn` where `a0..an` and `b0..bn` are the digits of the factors.
        // Each row of following matrices represents the multiplication of a digit of one factor
        // with each digit of the other factor, with offset of the first factors digit-position.

        // Matrix with: "digits_count" rows and "2*digits" columns
        let mut prod_rows = vec![vec![0 as DigitT; self.len() * 2]; self.len()];

        // `prod_carries[i][j]` represents the carry, which resulted from multiplying the i-th digit
        // of one factor by the (j-1)-th digit of the other factor. This carry must then be added to
        // the j-th digit of the end result.
        // First column of carries is always zero.
        let mut prod_carries = vec![vec![0 as DigitT; self.len() * 2]; self.len()];

        for i in 0..prod_rows.len() {
            let b_i = rhs[i] as DoubleDigitT;

            for j in 0..self.len() {
                let a_j = self[j] as DoubleDigitT;
                let prod_ij = a_j * b_i;

                prod_rows[i][j + i] = prod_ij as DigitT;
                prod_carries[i][j + i + 1] = (prod_ij >> DIGIT_BITS) as DigitT;
            }
        }

        // Sum columns

        let mut end_product = MPint::new(max_new_width);
        let mut col_carry = 0 as DigitT;
        for j in 0..prod_rows[0].len() {
            // Add last columns carry
            end_product += col_carry;

            for i in 0..prod_rows.len() {
                col_carry = end_product.carry_ripple_add_bins_inplace(&MPint::from_digit(
                prod_rows[i][j],
                max_new_width,
                )) as u64;

                col_carry += end_product.carry_ripple_add_bins_inplace(&MPint::from_digit(
                prod_carries[i][j],
                max_new_width,
                )) as u64;
            }

            // Make sure next column is added to the correct position of the number system
            end_product.rotate_digits_right(1);
        }

        // ~~~~ Epilog ~~~~
        end_product.trim_empty_end(self.len());
        end_product.sign = result_sign;

        end_product
    }
\end{lstlisting}