\section{Extras}\label{sec:extras}

\subsection{\ilc{rustdoc}}
Wie bereits erwähnt ist der Quellcode ausführlich mittels \emph{doc comment}s dokumentiert.
Mit folgendem Befehl lässt sich daraus eine HTML-Dokumentation in gewohnter ,,Rust-Optik'' generieren, welche im Anschluss unter \ilc{target/doc} zu finden ist:

\begin{lstlisting}[language=bash]
cargo doc --no-deps --examples --bins

# Alternativ: Inklusive privater Elemente
cargo doc --no-deps --examples --bins --document-private-items
\end{lstlisting}

\subsection{Demo CLI}
Dies ist die Standard-Anwendung, welche beim Aufruf von \ilc{cargo run} ausgeführt wird.
Es gibt zwei Modi:
\begin{itemize}
    \item Zum einen kann man eine Anzahl \textbf{zufälliger Testoperationen} ausführen und diese auf Wunsch inkl. des Ergebisses in einer Textdatei speichern lassen. Dabei ist das Format\footnote{Pro Zeile: \ilc{\{lhs\}\{operator\}\{rhs\}==\{mpa\_lib Ergebnis\}}} so gewählt, dass der komplette Inhalt direkt in eine Python Shell eingefügt werden kann, worauf automatisch alle Ergebnisse mit Pythons Ergebnissen verglichen werden.

    \item Zum anderen gibt es einen \textbf{interaktiven Modus}, der eine einfache REPL\footnote{read-evaluate-print-loop} bietet.
\end{itemize}

Eine ausführliche Beschreibung ist im Quellcode (\ilc{src/bin/mpa\_demo\_cli.rs}) bzw. als \ilc{rustdoc} (s. o.) verfügbar.

Vefügbare Optionen können zudem angezeigt werden mittels
\begin{lstlisting}[language=bash]
cargo run -- --help
\end{lstlisting}


\subsection{Examples}
Im Ordner \ilc{examples} sind folgende Beispiel Anwendungen der Bibliothek vorhanden:

\begin{minipage}{10cm}
    \dirtree{%
        .1 examples/.
        .2 binomial.rs.
        .2 exponentiation.rs.
        .2 simple\_factorial.rs.
    }
\end{minipage}

Um eines der Beispiele auszuführen kann man folgenden Befehl benutzen:

\begin{lstlisting}[language=bash]
cargo run --example binomial
\end{lstlisting}

Zum Teil beinhalten die Beispiele auch Unit-Tests, welche nicht zusammen mit den normalen Tests der Bibliothek aufgerufen werden. Sie können aber manuell wie folgt ausgeführt werden:

\begin{lstlisting}[language=bash]
cargo test --example binomial

# bzw. alle auf einmal:
cargo test --examples
\end{lstlisting}